%!TEX root = ../doc.tex
\chapter{Fazit}
\label{sec:fazit}
Das Entwickeln eines Prototypen mit dem geplanten Funktionsumfang hat sich als machbar herausgestellt. Die App Funktioniert zuverlässig und kann von normalen Benutzern gebraucht werden. Die beiden Use Cases (UC-001 Aktion definieren und UC-002 Aktion ausführen) sowie die 6 Funktionalen Anforderungen, welche im Kapitel \ref{sec:anforderungsanalyse} beschrieben wurden, wurden alle umgesetzt. \newline{} Es hat sich jedoch herausgestellt dass das Anbieten von verschiedenen Aktionen schwieriger ist als ursprünglich angenommen. Das aktivieren bzw. deaktivieren des GPS Moduls ist auf einem Android Mobiltelefon ohne Interaktion eines Benutzers nicht möglich. Auch für eine Aktion, welche es erlaubt eine Email zu verschicken muss bedeutend mehr Programmiert werden. Zum Beispiel müssen die aktiven Benutzerkonten geladen werden wofür wieder mehr Berechtigungen benötigt werden. Aber es gibt auch kleinere Verbesserungen welche noch nicht implementiert sind. So sollten die Aktionen auch in der Datenbank gespeichert werden anstatt in eine Datei serialisiert zu werden. Für die besser Stabilität sollten Tests definiert werden und vor einer Veröffentlichung sollte die App auf mehreren Geräten getestet werden. Das Projekt wird nach Abgabe dieser Seminararbeit weiter verfolgt werden.