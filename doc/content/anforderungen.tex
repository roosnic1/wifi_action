%!TEX root = ../doc.tex
\chapter{Anforderungsanalyse}
\label{sec:anforderungsanalyse}

\section{Use Cases}
Für die bessere Lesbarkeit werden in den Use Cases die folgenden Felder weg gelassen, weil sie immer den selben Wert haben.
\begin{usecase}
\addtitle{UC-000}{-}
\additemizedfield{Autoren}{
	\item (immer) Nicolas Roos
}
\additemizedfield{Quelle}{
	\item (immer) Nicolas Roos
}
\additemizedfield{Verantwortlicher}{
	\item (immer) Nicolas Roos
}

\end{usecase}

\newpage{}
\addcontentsline{toc}{subsection}{UC-001 Aktion definieren}
\begin{usecase}
\addtitle{UC-001}{Aktion definieren}
\addfield{Priorität}{Zwingend}
\addfield{Kritikalität}{Hoch}
\addfield{Beschreibung}{Der Benutzer kann eine Aktion für das erreichen bzw. verlassen eines bestimmten Wireless LANs definieren}
\addfield{Auslösendes Ereignis}{Der Benutzer erstellt eine neue Aktion}
\addfield{Akteure}{Benutzer}
\addfield{Vorbedingungen}{-}
\addfield{Nachbedingungen}{-}
\addfield{Ergebnis}{Eine gespeicherte Aktion}
\addscenario{Hauptszenario}{
	\item Der Benutzer erstellt eine neue Aktion.
	\item Der Benutzer wählt für die Aktion ein bestimmtes bekanntes Wireless LAN aus.
	\item Der Benutzer wählt wann die Aktion ausgeführt wird.
	\item Der Benutzer gibt noch zusätzliche Parameter ein.
	\item Die App speichert die Aktion.
}
\addfield{Alternativszenarien}{-}
%\addscenario{Alternativszenarien}{
%	\item Job wird über API gestartet...
%}
\addfield{Ausnahmeszenarien}{-}
%\addscenario{Ausnahmeszenarien}{\item ...}
\addfield{Qualitäten}{-}
\end{usecase}

\newpage{}
\addcontentsline{toc}{subsection}{UC-002 Aktion ausführen}
\begin{usecase}
\addtitle{UC-002}{Aktion ausführen}
\addfield{Priorität}{Zwingend}
\addfield{Kritikalität}{Hoch}
\addfield{Beschreibung}{Das System kann beim erreichen bzw. verlassen eines bestimmten Wireless LANs die definierte Aktion ausführen}
\addfield{Auslösendes Ereignis}{Die App erhält eine Meldung vom Android System}
\addfield{Akteure}{App}
\addfield{Vorbedingungen}{Die App läuft im Vordergrund oder Hintergrund und hat die Möglichkeit Systemmeldungen zu bekommen}
\addfield{Nachbedingungen}{-}
\addfield{Ergebnis}{Eine ausgeführte Aktion}
\addscenario{Hauptszenario}{
	\item Die App bekommt eine Meldung über die Änderung des Wireless LANs Status
	\item Die App überprüft ob für die Kombination (SSID und erreichen bzw. verlassen) eine Aktion vorhanden ist
	\item Die App führt die Aktion aus und Loggt das Ereignis
}
\addfield{Alternativszenarien}{-}
%\addscenario{Alternativszenarien}{
%	\item Job wird über API gestartet...
%}
\addfield{Ausnahmeszenarien}{-}
%\addscenario{Ausnahmeszenarien}{\item ...}
\addfield{Qualitäten}{-}
\end{usecase}


\newpage{}
\section{Funktionale Anforderungen}
\label{sec:funktionaleanforderungen}

\addcontentsline{toc}{subsection}{FREQ-001 Bekannte Wireless LANs auflisten}
\begin{usecase}
\addtitle{FREQ-001}{Bekannte Wireless LANs auflisten}
\addfield{UC-Referenz}{UC-001}
\addfield{Beschreibung}{Die App kann die dem Android System bekannten Wireless LANs auflisten. Bekannte Wireless LANs sind welche mit denen das Android System bereits eine Verbindung hergestellt hat. Von den bekannten Wireless LANs kann die APP die SSID lesen.}
\end{usecase}

\addcontentsline{toc}{subsection}{FREQ-002 Neue Aktion erstellen}
\begin{usecase}
\addtitle{FREQ-002}{Neue Aktion erstellen}
\addfield{UC-Referenz}{UC-001}
\addfield{Beschreibung}{Der Benutzer kann eine Aktion erstellen, dafür ein Wireless LAN auswählen und den Ausführungszeitpunk definieren. Die App kann die Aktion persistent abspeichern.}
\end{usecase}

\addcontentsline{toc}{subsection}{FREQ-003 Aktion löschen}
\begin{usecase}
\addtitle{FREQ-003}{Aktion löschen}
\addfield{UC-Referenz}{UC-001}
\addfield{Beschreibung}{Der Benutzer kann eine bereits erstellt Aktion löschen.}
\end{usecase}

\addcontentsline{toc}{subsection}{FREQ-004 Wirless LAN Änderungen erkennen}
\begin{usecase}
\addtitle{FREQ-004}{Wirless LAN Änderungen erkennen}
\addfield{UC-Referenz}{UC-002}
\addfield{Beschreibung}{Die App erkennt Änderungen des Wireless LANs status und kann die SSID des verbundenen Wireless LANs lesen.}
\end{usecase}

\addcontentsline{toc}{subsection}{FREQ-005 Aktion ausführen}
\begin{usecase}
\addtitle{FREQ-005}{Aktion ausführen}
\addfield{UC-Referenz}{UC-002}
\addfield{Beschreibung}{Die App kann die definierten Aktion ausführen. Die App muss dazu nicht aktiv genutzt werden.}
\end{usecase}

\addcontentsline{toc}{subsection}{FREQ-006 Ereignis loggen}
\begin{usecase}
\addtitle{FREQ-006}{Ereignis loggen}
\addfield{UC-Referenz}{UC-002}
\addfield{Beschreibung}{Die App kann die Ereignisse  Wireless LAN Status Änderungen und die ausgeführten Aktion loggen.}
\end{usecase}

