%!TEX root = ../doc.tex
\chapter{Anforderungsanalyse}
\label{sec:anforderungsanalyse}

\section{Use Cases}
Für die bessere Lesbarkeit werden in den Use Cases die folgenden Felder weg gelassen, weil sie immer den selben Wert haben.
\begin{usecase}
\addtitle{UC-000}{-}
\additemizedfield{Autoren}{
	\item (immer) Nicolas Roos
}
\additemizedfield{Quelle}{
	\item (immer) Nicolas Roos
}
\additemizedfield{Verantwortlicher}{
	\item (immer) Nicolas Roos
}

\end{usecase}

\newpage{}
\addcontentsline{toc}{subsection}{UC-001 Aktion definieren}
\begin{usecase}
\addtitle{UC-001}{Aktion definieren}
\addfield{Priorität}{Zwingend}
\addfield{Kritikalität}{Hoch}
\addfield{Beschreibung}{Der Benutzer kann eine Aktion für das erreichen bzw. verlassen eines bestimmten Wireless LANs definieren}
\addfield{Auslösendes Ereignis}{Der Benutzer erstellt eine neue Aktion}
\addfield{Akteure}{Benutzer}
\addfield{Vorbedingungen}{-}
\addfield{Nachbedingungen}{-}
\addfield{Ergebnis}{Eine gespeicherte Aktion}
\addscenario{Hauptszenario}{
	\item Der Benutzer erstellt eine neue Aktion.
	\item Der Benutzer wählt für die Aktion ein bestimmtes bekanntes Wireless LAN aus.
	\item Der Benutzer wählt wann die Aktion ausgeführt wird.
	\item Der Benutzer gibt noch zusätzliche Parameter ein.
	\item Die App speichert die Aktion.
}
\addfield{Alternativszenarien}{-}
%\addscenario{Alternativszenarien}{
%	\item Job wird über API gestartet...
%}
\addfield{Ausnahmeszenarien}{-}
%\addscenario{Ausnahmeszenarien}{\item ...}
\addfield{Qualitäten}{-}
\end{usecase}

\newpage{}
\addcontentsline{toc}{subsection}{UC-002 Aktion ausführen}
\begin{usecase}
\addtitle{UC-002}{Aktion ausführen}
\addfield{Priorität}{Zwingend}
\addfield{Kritikalität}{Hoch}
\addfield{Beschreibung}{Das System kann beim erreichen bzw. verlassen eines bestimmten Wireless LANs die definierte Aktion ausführen}
\addfield{Auslösendes Ereignis}{Die App erhält eine Meldung vom Android System}
\addfield{Akteure}{App}
\addfield{Vorbedingungen}{Die App läuft im Vordergrund oder Hintergrund und hat die Möglichkeit Systemmeldungen zu bekommen}
\addfield{Nachbedingungen}{-}
\addfield{Ergebnis}{Eine ausgeführte Aktion}
\addscenario{Hauptszenario}{
	\item Die App bekommt eine Meldung über die Änderung des Wireless LANs Status
	\item Die App überprüft ob für die Kombination (SSID und erreichen bzw. verlassen) eine Aktion vorhanden ist
	\item Die App führt die Aktion aus und Loggt das Ereignis
}
\addfield{Alternativszenarien}{-}
%\addscenario{Alternativszenarien}{
%	\item Job wird über API gestartet...
%}
\addfield{Ausnahmeszenarien}{-}
%\addscenario{Ausnahmeszenarien}{\item ...}
\addfield{Qualitäten}{-}
\end{usecase}


\newpage{}
\section{Funktionale Anforderungen}
\label{sec:funktionaleanforderungen}