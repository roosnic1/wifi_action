%!TEX root = ../doc.tex
\chapter{Einleitung}
\label{sec:Einleitung}

\section{Ausgangslage}
\label{sec:Ausgangslage}

Viele Mobile Apps erlauben dem Benutzer bzw. der Benutzerin Aktionen auszuführen, wenn ein bestimmter geographischer Punkt bzw. Bereich erreicht oder verlassen wird. Aufgrund des erhöhten Batterieverbrauchs bei aktivem GPS Dienst und der hohen Anzahl von Wireless LANs in den meisten grösseren Städten, erscheint es sinnvoll diese Benutzerdefinierten Aktionen auch beim erreichen oder verlassen eines Wireless LANs auszuführen.

\section{Ziel der Arbeit}
\label{sec:zielderarbeit}
Das Ziel der Arbeit ist eine Prototyp App für Android zu entwickeln, welche dem Benutzer bzw. der Benutzerin erlaubt Aktionen zu definieren, welche beim Verbinden bzw. Verlassen von Wireless LANs ausgeführt werden. Unter Aktionen wird zum Beispiel das ändern des Benachrichtigungsmodus (Lautlos, Haptisches Feedback, Klingelton) oder die Aktivierung bzw. Deaktivierung des Bluetooth Verbindung verstanden.

\section{Aufgabenstellung}
\label{sec:aufgabenstellung}
\begin{itemize}
  \item Recherche über bereits existierende Apps und die Möglichkeiten des Android SDK.
  \item Erstellen einer kleine Anforderungsanalyse.
  \item Entwicklung eines Prototyp der Android App.
  \item Erstellen eines Entwicklungsbericht.
  \item Erstellen einer Präsentation.
\end{itemize}

\section{Erwartete Resultate}
\label{sec:erwarteteresultate}
\begin{itemize}
  \item Abgeschlossene Recherche
  \item Fertige Anforderungsanalyse mit Anforderungen und Tests
  \item Getesteter und funktionierender Prototyp
  \item Entwicklungsbericht
  \item Präsentation
\end{itemize}